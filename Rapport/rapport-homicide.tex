\documentclass[a4paper,10pt]{report}
\usepackage[utf8]{inputenc}

% Title Page
\title{Homicides aux Etats-Unis}
\author{Vincent Bisogno - Vincent Gaillard - Ronan Desplanques - Roman Fenioux}


\begin{document}
\maketitle



\section{Introduction}


    The names of all project members ---ok\\
    The topic to be addressed, ---OK\\
    the datasets to use, where the data come from, and their formats. --- A TERMINER (format etc..)\\
    A description of the problem ---- A FAIRE\\
        Who are the users? --- A DEVELOPPER\\
        What are their backgrounds? --- A DEVELOPPER\\
        What are they trying to understand from the data? --- A FAIRE\\
        Is your visualization aimed primarily at exploring or communicating the data? \\ --- A FAIRE\\
    A description of the data --- OK\\
        What are the characteristics of the data, the attributes, the size of the dataset, etc. --- A TERMINER\\


\subsection{Public visé}
Notre visualisation est adressé au grand public : des personnes non qualifiées, sans connaisances particulières en statistique et intéressée
par le sujet. Nous voulons donc faire une représentation accessible et intuitive.

A COMPLETER


\subsection{Description du dataset}
Le dataset que nous utiliserons a été rassemblé par le Murder Accountability Project, fondé par Thomas Hargrove. C'est la base de données
la plus complète disponible actuellement aux Etats-Unis. Elle s'étend du Supplementary Homicide Report publié par le FBI en 1976 jusqu'à nos jours
et inclus des informations sur les circonstances du meurtre ainsi que l'âge, l'origine ethnique et le genre de la victime et du criminel ainsi que
leur éventuel lien et l'arme utilisée.

\end{document}          
